\documentclass[a4paper, 11pt,UTF8,oneside]{article}
\usepackage{ctex}
\usepackage{multirow,tabularx}
\usepackage{array,booktabs}
\usepackage{amsmath,dsfont,amsfonts,bm,amsbsy}
\usepackage[top=0.5in, bottom=0.5in, left=0.5in, right=0.5in]{geometry}
%\usepackage{lstlisting} % Write codes.
\usepackage[dvipsnames]{xcolor}
\usepackage{enumerate}
\usepackage{tikz}
\usepackage{etoolbox}
\usepackage{listings}
\usepackage{mdframed}
\usetikzlibrary{cd}
\usetikzlibrary{graphs}
\usetikzlibrary {positioning}
\definecolor {processblue}{cmyk}{0.96,0,0,0}
\usetikzlibrary{shapes.multipart}
%\usetikzlibrary{graphdrawing}
%\usegdlibrary{layered}
%Differential operator.
\newcommand*\dif{\mathop{}\!\mathrm{d}}
%Circled number.
\newcommand{\circled}[2][]{%
  \tikz[baseline=(char.base)]{%
    \node[shape = circle, draw, inner sep = 1pt]
    (char) {\phantom{\ifblank{#1}{#2}{#1}}};%
    \node at (char.center) {\makebox[0pt][c]{#2}};}}
\robustify{\circled}

\usepackage{titlesec,color,blindtext} % To use chapter.
\definecolor{gray75}{gray}{0.75}
\newcommand{\hsp}{\hspace{20pt}}
\titleformat{\chapter}[hang]{\Huge\bfseries}{\thechapter\hsp\textcolor{gray75}{|}\hsp}{0pt}{\Huge\bfseries}

%Highlight V2
\usepackage{tcolorbox}
\newtcbox{\entoure}[1][red]{on line,
arc=3pt,colback=#1!10!white,colframe=#1!50!black,
before upper={\rule[-3pt]{0pt}{10pt}},boxrule=1pt,
boxsep=0pt,left=2pt,right=2pt,top=1pt,bottom=.5pt}

\usepackage{color}   %May be necessary if you want to color links
\usepackage{hyperref}
\hypersetup{
    colorlinks=true, %set true if you want colored links
    linktoc=all,     %set to all if you want both sections and subsections linked
    linkcolor=blue,  %choose some color if you want links to stand out
}
\usepackage{longtable}
\usepackage{multirow}
\usepackage{multicol}
\lstset{
%numbers=left,
%numberstyle=\small,
%numbersep=8pt,
%frame = single,
language=C++,
framexleftmargin=15pt,
basicstyle=\ttfamily,
commentstyle=\color{green}, % comment color
keywordstyle=\color{blue}, % keyword color
stringstyle=\color{red} % string color
}
\newcommand{\cd}[1]{\texttt{#1}}
\usepackage{hyphenat}


\begin{document}
\title{tusharecpp参考手册}
\author{周瑶}
\maketitle

\section{安装}
\paragraph{依赖项}依赖项主要是boost的几个库,如果已经安装了boost,可以直接使用。
\begin{enumerate}
  \item \cd{boost.beast}。C++网络库
  \item \cd{boost.property\_tree}。用于处理JSON数据。
  \item \cd{boost.tokenizer}。分割字符串。
\end{enumerate}

\cd{tusharecpp}是一个纯头文件的C++库,将文件夹内的"tusharecpp"文件夹整个复制到可以让编译器找到的地方就可以使用。接口与tushare的python接口基本一致。需要编译器支持C++17或以上,在VS中用时,需要使用\cd{$\backslash$std:c++latest}。

\section{基本使用}
详细的接口参考\href{https://tushare.pro/document/2}{tushare官网}。

C++代码:

\begin{tcolorbox}
\begin{lstlisting}
#include <iostream>
#include "../../../tusharecpp/tusharecpp.hpp"
using namespace std;

int main()
{
    auto pro = ts::pro_api("xxxxxxxxxxxxxxxxxxxxxxx");
    auto data=pro.stock_basic("","","");
	cout << "Begin write;" << endl;
    data.to_csv("data.csv");
    std::cout << "Hello World!\n";
}
\end{lstlisting}
\end{tcolorbox}

其中\cd{pro\_api}需要替换成自己的token,这个token要从tushare取得。

\section{设计原理}
\paragraph{代码结构}整个库只由两个文件构成:\cd{data\_container.hpp}和\cd{tusharecpp.hpp}。每个接口要接受特定的参数,参数名是直接从tushare网站上抓取,然后使用代码生成的方法产生的,见于\cd{code\_generate}文件夹内的nb文件,而模板文件是该文件夹内的\cd{tusharecpp.hpp}。

模板文件内有\cd{\#\#\_AUTO\_GENERATED\_FIELDS\_MAP\_\#\#}、\cd{\#\#\_AUTO\_GENERATED\_TYPES\_MAP\_\#\#}、\cd{\#\#\_AUTO\_GENERATED\_API\_\#\#}几个串,它们会被替换掉。\cd{fields\_map}中存储的每个接口返回的fields,\cd{types\_map}存储的是返回的每个变量的类型。

\paragraph{数据结构}\entoure{data\_container}是用来存储返回的数据的,内部用\cd{vector<void*>}存储每列的指针,在需要时再\cd{reinterpret\_cast}回去。因此支持的数据类型有限,目前有三种:\cd{int}、\cd{string}、\cd{double}。

\section{API说明}
这个库在写的时候没有考虑多线程,也不建议使用多线程,因为已经具有了爬虫的性质,如果给服务器的负载过大,可能有法律上的问题,再一个也影响其它人使用。


\end{document} 